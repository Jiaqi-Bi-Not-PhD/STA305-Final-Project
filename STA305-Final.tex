\PassOptionsToPackage{unicode=true}{hyperref} % options for packages loaded elsewhere
\PassOptionsToPackage{hyphens}{url}
\PassOptionsToPackage{dvipsnames,svgnames*,x11names*}{xcolor}
%
\documentclass[12pt,]{article}
\usepackage{lmodern}
\usepackage{amssymb,amsmath}
\usepackage{ifxetex,ifluatex}
\usepackage{fixltx2e} % provides \textsubscript
\ifnum 0\ifxetex 1\fi\ifluatex 1\fi=0 % if pdftex
  \usepackage[T1]{fontenc}
  \usepackage[utf8]{inputenc}
  \usepackage{textcomp} % provides euro and other symbols
\else % if luatex or xelatex
  \usepackage{unicode-math}
  \defaultfontfeatures{Ligatures=TeX,Scale=MatchLowercase}
    \setmainfont[]{Times New Roman}
\fi
% use upquote if available, for straight quotes in verbatim environments
\IfFileExists{upquote.sty}{\usepackage{upquote}}{}
% use microtype if available
\IfFileExists{microtype.sty}{%
\usepackage[]{microtype}
\UseMicrotypeSet[protrusion]{basicmath} % disable protrusion for tt fonts
}{}
\IfFileExists{parskip.sty}{%
\usepackage{parskip}
}{% else
\setlength{\parindent}{0pt}
\setlength{\parskip}{6pt plus 2pt minus 1pt}
}
\usepackage{xcolor}
\usepackage{hyperref}
\hypersetup{
            pdftitle={ANOVA, Model Assumption and Statistical Analysis on Effect of Household Processing Method to pH Value of Water},
            pdfauthor={Jiaqi Bi, 1003886609; Lei Cao, 1005715111; Lanruo Li, 1005149581; Le Shen, 1005935106; Yuika Cho, 1003213186},
            colorlinks=true,
            linkcolor=Maroon,
            filecolor=Maroon,
            citecolor=Blue,
            urlcolor=blue,
            breaklinks=true}
\urlstyle{same}  % don't use monospace font for urls
\usepackage[margin=2.5cm]{geometry}
\usepackage{color}
\usepackage{fancyvrb}
\newcommand{\VerbBar}{|}
\newcommand{\VERB}{\Verb[commandchars=\\\{\}]}
\DefineVerbatimEnvironment{Highlighting}{Verbatim}{commandchars=\\\{\}}
% Add ',fontsize=\small' for more characters per line
\usepackage{framed}
\definecolor{shadecolor}{RGB}{248,248,248}
\newenvironment{Shaded}{\begin{snugshade}}{\end{snugshade}}
\newcommand{\AlertTok}[1]{\textcolor[rgb]{0.94,0.16,0.16}{#1}}
\newcommand{\AnnotationTok}[1]{\textcolor[rgb]{0.56,0.35,0.01}{\textbf{\textit{#1}}}}
\newcommand{\AttributeTok}[1]{\textcolor[rgb]{0.77,0.63,0.00}{#1}}
\newcommand{\BaseNTok}[1]{\textcolor[rgb]{0.00,0.00,0.81}{#1}}
\newcommand{\BuiltInTok}[1]{#1}
\newcommand{\CharTok}[1]{\textcolor[rgb]{0.31,0.60,0.02}{#1}}
\newcommand{\CommentTok}[1]{\textcolor[rgb]{0.56,0.35,0.01}{\textit{#1}}}
\newcommand{\CommentVarTok}[1]{\textcolor[rgb]{0.56,0.35,0.01}{\textbf{\textit{#1}}}}
\newcommand{\ConstantTok}[1]{\textcolor[rgb]{0.00,0.00,0.00}{#1}}
\newcommand{\ControlFlowTok}[1]{\textcolor[rgb]{0.13,0.29,0.53}{\textbf{#1}}}
\newcommand{\DataTypeTok}[1]{\textcolor[rgb]{0.13,0.29,0.53}{#1}}
\newcommand{\DecValTok}[1]{\textcolor[rgb]{0.00,0.00,0.81}{#1}}
\newcommand{\DocumentationTok}[1]{\textcolor[rgb]{0.56,0.35,0.01}{\textbf{\textit{#1}}}}
\newcommand{\ErrorTok}[1]{\textcolor[rgb]{0.64,0.00,0.00}{\textbf{#1}}}
\newcommand{\ExtensionTok}[1]{#1}
\newcommand{\FloatTok}[1]{\textcolor[rgb]{0.00,0.00,0.81}{#1}}
\newcommand{\FunctionTok}[1]{\textcolor[rgb]{0.00,0.00,0.00}{#1}}
\newcommand{\ImportTok}[1]{#1}
\newcommand{\InformationTok}[1]{\textcolor[rgb]{0.56,0.35,0.01}{\textbf{\textit{#1}}}}
\newcommand{\KeywordTok}[1]{\textcolor[rgb]{0.13,0.29,0.53}{\textbf{#1}}}
\newcommand{\NormalTok}[1]{#1}
\newcommand{\OperatorTok}[1]{\textcolor[rgb]{0.81,0.36,0.00}{\textbf{#1}}}
\newcommand{\OtherTok}[1]{\textcolor[rgb]{0.56,0.35,0.01}{#1}}
\newcommand{\PreprocessorTok}[1]{\textcolor[rgb]{0.56,0.35,0.01}{\textit{#1}}}
\newcommand{\RegionMarkerTok}[1]{#1}
\newcommand{\SpecialCharTok}[1]{\textcolor[rgb]{0.00,0.00,0.00}{#1}}
\newcommand{\SpecialStringTok}[1]{\textcolor[rgb]{0.31,0.60,0.02}{#1}}
\newcommand{\StringTok}[1]{\textcolor[rgb]{0.31,0.60,0.02}{#1}}
\newcommand{\VariableTok}[1]{\textcolor[rgb]{0.00,0.00,0.00}{#1}}
\newcommand{\VerbatimStringTok}[1]{\textcolor[rgb]{0.31,0.60,0.02}{#1}}
\newcommand{\WarningTok}[1]{\textcolor[rgb]{0.56,0.35,0.01}{\textbf{\textit{#1}}}}
\usepackage{graphicx,grffile}
\makeatletter
\def\maxwidth{\ifdim\Gin@nat@width>\linewidth\linewidth\else\Gin@nat@width\fi}
\def\maxheight{\ifdim\Gin@nat@height>\textheight\textheight\else\Gin@nat@height\fi}
\makeatother
% Scale images if necessary, so that they will not overflow the page
% margins by default, and it is still possible to overwrite the defaults
% using explicit options in \includegraphics[width, height, ...]{}
\setkeys{Gin}{width=\maxwidth,height=\maxheight,keepaspectratio}
\setlength{\emergencystretch}{3em}  % prevent overfull lines
\providecommand{\tightlist}{%
  \setlength{\itemsep}{0pt}\setlength{\parskip}{0pt}}
\setcounter{secnumdepth}{0}
% Redefines (sub)paragraphs to behave more like sections
\ifx\paragraph\undefined\else
\let\oldparagraph\paragraph
\renewcommand{\paragraph}[1]{\oldparagraph{#1}\mbox{}}
\fi
\ifx\subparagraph\undefined\else
\let\oldsubparagraph\subparagraph
\renewcommand{\subparagraph}[1]{\oldsubparagraph{#1}\mbox{}}
\fi

% set default figure placement to htbp
\makeatletter
\def\fps@figure{htbp}
\makeatother

\usepackage{float}
\let\origfigure\figure
\let\endorigfigure\endfigure
\renewenvironment{figure}[1][2] {
    \expandafter\origfigure\expandafter[H]
} {
    \endorigfigure
}
\usepackage{lastpage}
\usepackage{fancyhdr}
\usepackage{setspace}
\usepackage{float}
\singlespacing
\pagestyle{fancy}
\fancyhead[CO, CE]{Group 22}
\fancyhead[LE, RO]{STA305 Final Report}
\fancyfoot[CO, CE]{\thepage \ of \pageref{LastPage}}
\floatplacement{figure}{H}
\usepackage{etoolbox}
\makeatletter
\providecommand{\subtitle}[1]{% add subtitle to \maketitle
  \apptocmd{\@title}{\par {\large #1 \par}}{}{}
}
\makeatother

\title{\textbf{ANOVA, Model Assumption and Statistical Analysis on Effect of
Household Processing Method to pH Value of Water}}
\providecommand{\subtitle}[1]{}
\subtitle{\textbf{STA305 Final Report (LEC 0101), Group 22}}
\author{Jiaqi Bi, 1003886609 \and Lei Cao, 1005715111 \and Lanruo Li, 1005149581 \and Le Shen, 1005935106 \and Yuika Cho, 1003213186}
\date{}

\begin{document}
\maketitle

{
\hypersetup{linkcolor=}
\setcounter{tocdepth}{3}
\tableofcontents
}
\newpage

\hypertarget{introduction}{%
\subsection{Introduction}\label{introduction}}

The increased concern on water quality has been rigorously demanding,
and this has been a public health issue for years. Primarily there are
miscellaneous scientific articles and news about the pH value of water
affects human health. The question of interest lies in if the household
water processing method will affect the pH value. The experiment uses
the most common approach that everyone can practice at home. After
determining the sample size and the collection of experiment data,
one-way ANOVA, linear regression modelling with dummy coding method, and
contrasts analysis were practiced to check the statistical significance
of the assumption. The main purpose of the project is to find the best
way of producing an ideally higher pH value.

\hypertarget{experimental-design-and-process}{%
\subsection{Experimental Design and
Process}\label{experimental-design-and-process}}

The household water processing methods include Stilled Water, Boiled
Water, Filtered Water, and Frozen Water. The water source is controlled
to be tap water as gaining water from the tap is widely habitual
everywhere. Controlling the water source prevents excessive biases
caused by different unforeseeable water purity. Moreover, the initial
volume of the cup of water is controlled. Indeed, the experimental unit
is the cup of water. To reduce the confoundings of the unit, a
Between-Subjects design is practiced throughout. As mentioned, there are
4 levels of the predictor variable. Due to the filling time of the water
may be different and hard to be controlled or blocked, using
randomization that after filling all cups of water by labeling each cup
randomly is the best way to minimize the ``time effect'' of output. The
repliaction is easy to follow that the water source could be selected
otherwise. There are total 48 cups of water with each has a unique
number, such that label 1-12 cups are given the treatment of being
frozen, 13-24 to be filtered, 25-36 to be boiled, and 37-48 to be
stilled (Control Group).

\begin{figure}[H]
  \centering
  \includegraphics[width=0.6\textwidth]{experiment photo.jpeg}
  \caption{Experiment Conduction with pH Meter and Thermometer}
\end{figure}
\newpage

The stilled water treatment can be treated as a control group to avoid
the time being a factor of the pH value change. Blindness is not
essential for this experiment as the output of pH value is examined by a
specific meter (objective measurement). There is another meter involved
in the experiment: the thermometer to ensure each treatment level has
the same initial temperature, and each unit of a level has the same
temperature after the treatment. Figure 1 shows these two testers and
the experiment conduction.

Specific treatment details:

\begin{itemize}
\item Stilled Water: Control Group, stilling the group until the end of the experiment.
\item Filtered Water: Using the same water filtration device to filter all cups of this group. 
\item Frozen Water: Using the same fridge to freeze all cups of this group simultaneously. 
\item Boiled Water: Using the same teakettle to boil all cups of this group simultaneously.. 
\end{itemize}

Specific experiment flowchart:

\begin{figure}[H]
  \centering
  \includegraphics[width=0.69\textwidth]{flowchart.png}
\end{figure}

\hypertarget{data-analysis}{%
\subsection{Data Analysis}\label{data-analysis}}

\hypertarget{sample-determination}{%
\subsubsection{Sample Determination}\label{sample-determination}}

Before the conduction of the experiment, the determination of sample
size is finalized by using Balanced one way ANOVA power calculation:
Manually set the power to be \(f=0.8\) for a large effect, that the
result of sample size determination turns out to be \(n\doteq 12\).

\hypertarget{data-visualization}{%
\subsubsection{Data Visualization}\label{data-visualization}}

Figure 2 is a general frame of the collected data where the data
visualization is practiced. Instead of showing the pH value output, the
graph illustrates the pH value difference of before and after experiment
for each treatment.

\begin{figure}
  \centering
  \includegraphics[width=0.8\textwidth]{"figure.1.jpg}
  \caption{Boxplot of Difference in pH per Water Processing Method}
\end{figure}

\hypertarget{mean-value-and-statistical-modeling}{%
\subsubsection{Mean Value and Statistical
Modeling}\label{mean-value-and-statistical-modeling}}

While a boxplot shows intuitionistic results, the statistical
significance of the result needs a deeper analysis. The mean value and
variance of each treatment show that:

\begin{table}[H]
\caption{Table of Mean Value and Variance of Data}
\vspace{5pt}
\centering
\begin{tabular}{ |p{3cm}||p{3cm}|p{3cm}| }
 \hline
 Treatment& Mean Value & Variance \\
 \hline
 Bolied Water   & $+0.842$    & $0.0081$\\
 Filtered Water &   $-0.675$  & $0.0166$   \\
 Frozen Water &$-0.225$ & $0.0130$\\
 Stilled Water    &$+0.075$ & $0.0111$\\
 \hline
\end{tabular}
\end{table}

The linear regression of the data using dummy coding scheme:
\[\hat{Y}_i=0.0750+0.7667I_{Boiled,i}-0.7500I_{Filtered,i}-0.3000I_{Frozen,i}\]
For \(m=Boiled, Filtered, Frozen\) as Stilled Water is the default and
has no indicators, \[
I_{m,i}=
\begin{cases}
1, \text{ if $i$th case is in level $m$ }\\
0, \text{ otherwise}
\end{cases}
\]

\hypertarget{one-way-anova-and-testing-contrasts}{%
\subsubsection{One-Way ANOVA and Testing
Contrasts}\label{one-way-anova-and-testing-contrasts}}

The null hypothesis is all mean pH value differences of 4 water
processing methods are the same. The alternative hypothesis is at least
one pair of mean pH value differences are not the same. The result shows
a significance (p-value \(<0.05\)) of the F-test. That is, the null
hypothesis is rejected. Thus, there is a strong evidence to conclude
that different water processing methods defer in pH value difference.

The formula of linear contrast is written as a linear combination of the
group means (mean difference in pH value): \[\psi=\sum_{j=1}^4c_j\mu_j\]
where \(j\) indicates each level, \(c_j\) is the coefficients in the
contrast that needs to be restricted as the sum of coefficients to be
\(0\): \[\sum_{j=1}^4c_j=0\]

The experiment contains 4 types of water processing, and it has 3
temperature levels after assigning treatment to those samples. The
temperature level is marked as ``Medium'' when the temperature after the
experiment is greater than 15 and less than 45, ``Low'' when the
temperature after the experiment is less than 15, ``High'' when the
temperature afterr the experiment is greater or equal to 45. Table 2
shows everry linear combination of group means and all comparisons.

\begin{table}[H]
\caption{Table of All Comparisons and P-value}
\vspace{5pt}
\centering
\begin{tabular}{ |c||c|c|c|c||c|c|  }
 \hline
 Group & Stilled & Boiled & Filtered & Frozen & Contrast Estimate & p-value\\
 \hline
 Temp Level & Medium & High & Medium & Low & & \\
 Means ($\bar{y_j}$) & $0.075$ & $0.8417$ & $-0.6750$ & $-0.2250$ & $\hat{\psi}$ & \\
 \hline
 $\psi_1$ & $1$ & $-1$ & $0$ & $0$ & $-0.7667$ & $<2\times 10^{16}$\\
 $\psi_2$ & $0$ & $1$ & $-1$ & $0$ & $1.5167$ & $<2\times 10^{16}$\\
 $\psi_3$ & $0$ & $0$ & $1$ & $-1$ & $-0.4500$ & $7.09\times 10^{13}$\\
 $\psi_4$ & $1$ & $-2$ & $1$ & $0$ & $-2.2834$ & $<2\times 10^{16}$\\
 $\psi_5$ & $1$ & $0$ & $1$ & $-2$ & $-0.1500$ & $0.0613$\\
 $\psi_6$ & $0$ & $1$ & $0$ & $-1$ & $1.0667$ & $<2\times 10^{16}$\\
 \hline
\end{tabular}
\end{table}

\hypertarget{type-i-error-solution}{%
\subsubsection{Type I Error Solution}\label{type-i-error-solution}}

There are total 6 tests with \(\alpha=0.05\):
\[\alpha_{FW}=P(\text{at least one Type I Error})=1-(1-0.05)^6=0.2649\]
The probability of having a Type I Error somewhere during tests is
\(26.49\%\). The test focuses on the pairwise comparison, and the
experiment design is balanced. Using Tukey's HSD correction is the best
way to solve high Type I error issue. There are \({4\choose 2}=6\) pairs
of comparison in the Tukey's HSD. Table 3 shows the 95\% family-wise
confidence interval level.

\begin{table}[H]
\caption{Tukey Multiple Comparisons of Means with $95\%$ Family-Wise CI}
\vspace{4pt}
\centering
\begin{tabular}{ |p{5cm}||p{4cm}|p{4cm}|  }
 \hline
 Treatment Comparisons & Confidence Interval&$\hat{\psi}$\\
 \hline
 Boiled vs. Stilled & $(0.6463,0.8870)$ &$0.7667$\\
 Filtered vs. Stilled & $(-0.8704,-0.6296)$ & $-0.7500$\\
 Frozen vs. Stilled & $(-0.4204,-0.1796)$ & $-0.3000$\\
 Filtered vs. Boiled & $(-1.6370,-1.3963)$ & $-1.5167$\\
 Frozen vs. Boiled & $(-1.1870,-0.9463)$ & $-1.0667$\\
 Frozen vs. Filtered & $(0.3296,0.5704)$ & $0.4500$\\
 \hline
\end{tabular}
\end{table}

\hypertarget{conclusion}{%
\subsection{Conclusion}\label{conclusion}}

Based on the complex comparison, there is strong evidence that Boiled
Water increases the pH value comparing to Stilled Water and Filtered
Water. There is also strong evidence that Filtered Water decreases pH
value comparing to Frozen Water, Higher temperature method increases pH
value comparing to Low and Medium temperature level. However, the
statistics show there is no evidence that Medium temperature decrease pH
value comparing to Low temperature. Furthermore, the Tukey's CI reduces
the Type I Error rate and gives a significant results of pairwise
comparisons since all comparisons do not contain \(0\) value of CI.
Thus, there is a strong evidence to deduce that different water
processing techniques do change the pH values accordingly. The best way
to increase pH value includes boiling the water with higher
temperatures, and the best way to decrease pH value is to filtering the
water while lowering the temperature.

\hypertarget{limitations-and-discussion}{%
\subsection{Limitations and
Discussion}\label{limitations-and-discussion}}

The experiment only takes place with the object water, other substances
are not considered. The temperature indeed plays a strong role in
altering the pH value. There are more methods to consider the
temperature as a factor, such as ANCOVA, or other variable selection
criteria that may be better statistical methods to deal with
quantitative factors. The experiment could have done better if there is
one more water source to be chosen from. In this manner, Two-Way ANOVA
can be developed and it expands the experiment details and diversity.
\newpage

\hypertarget{appendix}{%
\subsection{Appendix}\label{appendix}}

The appendix contains all R codes and outputs that used during the
experiment.

\begin{Shaded}
\begin{Highlighting}[]
\KeywordTok{library}\NormalTok{(pwr)}\CommentTok{# Packages used in the data analysis}
\KeywordTok{library}\NormalTok{(tidyverse)}\CommentTok{# Packages used in the data analysis}
\KeywordTok{library}\NormalTok{(tidyr)}\CommentTok{# Packages used in the data analysis}
\KeywordTok{library}\NormalTok{(ggplot2)}\CommentTok{# Packages used in the data analysis}
\KeywordTok{library}\NormalTok{(multcomp)}\CommentTok{# Packages used in the data analysis}
\KeywordTok{library}\NormalTok{(readxl)}\CommentTok{# Packages used in the data analysis}
\end{Highlighting}
\end{Shaded}

\begin{Shaded}
\begin{Highlighting}[]
\KeywordTok{pwr.anova.test}\NormalTok{(}\DataTypeTok{k=}\DecValTok{4}\NormalTok{, }\DataTypeTok{n=}\OtherTok{NULL}\NormalTok{, }\DataTypeTok{f=}\FloatTok{0.5}\NormalTok{, }\DataTypeTok{sig.level=}\FloatTok{0.05}\NormalTok{, }\DataTypeTok{power=}\FloatTok{0.8}\NormalTok{) }
\end{Highlighting}
\end{Shaded}

\begin{verbatim}
## 
##      Balanced one-way analysis of variance power calculation 
## 
##               k = 4
##               n = 11.92611
##               f = 0.5
##       sig.level = 0.05
##           power = 0.8
## 
## NOTE: n is number in each group
\end{verbatim}

\begin{Shaded}
\begin{Highlighting}[]
\CommentTok{#Sample Size Determination}
\end{Highlighting}
\end{Shaded}

\begin{Shaded}
\begin{Highlighting}[]
\NormalTok{Final.Data <-}\StringTok{ }\KeywordTok{read_excel}\NormalTok{(}\StringTok{"STA305 FINAL PROJECT DATA.xlsx"}\NormalTok{) }\CommentTok{#Read excel into R}
\NormalTok{Final.Data <-}\StringTok{ }\NormalTok{Final.Data[}\OperatorTok{-}\KeywordTok{c}\NormalTok{(}\DecValTok{49}\OperatorTok{:}\DecValTok{55}\NormalTok{),] }
\CommentTok{#Delete unnecessary rows}
\NormalTok{Final.Data <-}\StringTok{ }\NormalTok{Final.Data }\OperatorTok
\StringTok{  }\KeywordTok{rename}\NormalTok{(}
    \DataTypeTok{pH.Before=}\StringTok{`}\DataTypeTok{pH Before Treatment (0-14)}\StringTok{`}\NormalTok{,}
    \DataTypeTok{pH.After=}\StringTok{`}\DataTypeTok{pH After Treatment (0-14)}\StringTok{`}\NormalTok{,}
    \DataTypeTok{Temp.Before=}\StringTok{`}\DataTypeTok{Initial Temperature (℃)}\StringTok{`}\NormalTok{,}
    \DataTypeTok{Temp.After=}\StringTok{`}\DataTypeTok{Temperature After Treatment (℃)}\StringTok{`}
\NormalTok{  ) }\OperatorTok
\StringTok{  }\KeywordTok{mutate}\NormalTok{(Final.Data,}\DataTypeTok{Temp.After.Level=}
      \KeywordTok{ifelse}\NormalTok{(Temp.After}\OperatorTok{>=}\DecValTok{15} \OperatorTok{&}\NormalTok{Temp.After}\OperatorTok{<}\DecValTok{45}\NormalTok{,}\StringTok{"Medium"}\NormalTok{,}
      \KeywordTok{ifelse}\NormalTok{(Temp.After}\OperatorTok{<}\DecValTok{15}\NormalTok{, }\StringTok{"Low"}\NormalTok{, }
      \KeywordTok{ifelse}\NormalTok{(Temp.After}\OperatorTok{>=}\DecValTok{45}\NormalTok{, }\StringTok{"High"}\NormalTok{, }\StringTok{"NA"}\NormalTok{))))}
\CommentTok{#Data Wrangling, to record Low Medium High Temperatures into dataset}
\NormalTok{Final.Data}\OperatorTok{$}\NormalTok{DiffinPH=Final.Data}\OperatorTok{$}\NormalTok{pH.After}\OperatorTok{-}\NormalTok{Final.Data}\OperatorTok{$}\NormalTok{pH.Before}
\CommentTok{#Calculate pH differences}
\KeywordTok{attach}\NormalTok{(Final.Data)}
\CommentTok{#Attach the dataset}
\NormalTok{DiffinPH}\FloatTok{.1}\NormalTok{<-pH.After}\OperatorTok{-}\NormalTok{pH.Before}
\CommentTok{#Calculate pH differences}
\NormalTok{DiffinPH.Frozen <-}\StringTok{ }\NormalTok{DiffinPH[Treatment}\OperatorTok{==}\StringTok{"Frozen Water"}\NormalTok{]}
\CommentTok{#Calculate pH differences for Frozen Water}
\NormalTok{DiffinPH.Filtered <-}\StringTok{ }\NormalTok{DiffinPH[Treatment}\OperatorTok{==}\StringTok{"Filtered Water"}\NormalTok{]}
\CommentTok{#Calculate pH differences for Filtered Water}
\NormalTok{DiffinPH.Boiled <-}\StringTok{ }\NormalTok{DiffinPH[Treatment}\OperatorTok{==}\StringTok{"Boiled Water"}\NormalTok{]}
\CommentTok{#Calculate pH differences for Boiled Water}
\NormalTok{DiffinPH.Control <-}\StringTok{ }\NormalTok{DiffinPH[Treatment}\OperatorTok{==}\StringTok{"Control Group"}\NormalTok{]}
\CommentTok{#Calculate pH differences for Stilled Water}
\end{Highlighting}
\end{Shaded}

\begin{Shaded}
\begin{Highlighting}[]
\KeywordTok{ggplot}\NormalTok{(Final.Data, }\KeywordTok{aes}\NormalTok{(}\DataTypeTok{x=}\NormalTok{Treatment, }\DataTypeTok{y=}\NormalTok{DiffinPH, }\DataTypeTok{fill=}\NormalTok{Treatment))}\OperatorTok{+}
\StringTok{  }\KeywordTok{geom_boxplot}\NormalTok{()}\OperatorTok{+}
\StringTok{  }\KeywordTok{labs}\NormalTok{(}\DataTypeTok{title=}\StringTok{"Difference in pH per Water Processing Method"}\NormalTok{,}
       \DataTypeTok{y=}\StringTok{"PH Difference"}\NormalTok{)}\OperatorTok{+}
\StringTok{  }\KeywordTok{theme_minimal}\NormalTok{()}
\CommentTok{#Plot the boxplot}
\end{Highlighting}
\end{Shaded}

\begin{Shaded}
\begin{Highlighting}[]
\KeywordTok{tapply}\NormalTok{(DiffinPH}\FloatTok{.1}\NormalTok{, Treatment, mean) }\CommentTok{#Calculate pH Difference means}
\end{Highlighting}
\end{Shaded}

\begin{verbatim}
##   Boiled Water  Control Group Filtered Water   Frozen Water 
##      0.8416667      0.0750000     -0.6750000     -0.2250000
\end{verbatim}

\begin{Shaded}
\begin{Highlighting}[]
\KeywordTok{tapply}\NormalTok{(DiffinPH}\FloatTok{.1}\NormalTok{, Treatment, var)}\CommentTok{#Calculate pH Difference variance}
\end{Highlighting}
\end{Shaded}

\begin{verbatim}
##   Boiled Water  Control Group Filtered Water   Frozen Water 
##    0.008106061    0.011136364    0.016590909    0.012954545
\end{verbatim}

\begin{Shaded}
\begin{Highlighting}[]
\NormalTok{Final.Data}\OperatorTok{$}\NormalTok{Treatment <-}\StringTok{ }\KeywordTok{relevel}\NormalTok{(}\KeywordTok{factor}\NormalTok{(Final.Data}\OperatorTok{$}\NormalTok{Treatment), }\StringTok{"Control Group"}\NormalTok{)}
\CommentTok{# Set Stilled Water to be the default}
\NormalTok{model.final <-}\StringTok{ }\KeywordTok{lm}\NormalTok{(DiffinPH}\OperatorTok{~}\NormalTok{Treatment, }\DataTypeTok{data=}\NormalTok{Final.Data)}
\CommentTok{# Statistical Modeling for linear regression}
\KeywordTok{summary}\NormalTok{(model.final)}
\end{Highlighting}
\end{Shaded}

\begin{verbatim}
## 
## Call:
## lm(formula = DiffinPH ~ Treatment, data = Final.Data)
## 
## Residuals:
##      Min       1Q   Median       3Q      Max 
## -0.27500 -0.07500 -0.02500  0.05833  0.27500 
## 
## Coefficients:
##                         Estimate Std. Error t value Pr(>|t|)    
## (Intercept)              0.07500    0.03188   2.352   0.0232 *  
## TreatmentBoiled Water    0.76667    0.04509  17.004  < 2e-16 ***
## TreatmentFiltered Water -0.75000    0.04509 -16.635  < 2e-16 ***
## TreatmentFrozen Water   -0.30000    0.04509  -6.654 3.68e-08 ***
## ---
## Signif. codes:  0 '***' 0.001 '**' 0.01 '*' 0.05 '.' 0.1 ' ' 1
## 
## Residual standard error: 0.1104 on 44 degrees of freedom
## Multiple R-squared:  0.9646, Adjusted R-squared:  0.9622 
## F-statistic: 400.2 on 3 and 44 DF,  p-value: < 2.2e-16
\end{verbatim}

\begin{Shaded}
\begin{Highlighting}[]
\CommentTok{# To see details of linear regression model}
\end{Highlighting}
\end{Shaded}

\begin{Shaded}
\begin{Highlighting}[]
\KeywordTok{anova}\NormalTok{(model.final)}\CommentTok{#Run One-Way ANOVA of the model}
\end{Highlighting}
\end{Shaded}

\begin{verbatim}
## Analysis of Variance Table
## 
## Response: DiffinPH
##           Df  Sum Sq Mean Sq F value    Pr(>F)    
## Treatment  3 14.6425  4.8808  400.17 < 2.2e-16 ***
## Residuals 44  0.5367  0.0122                      
## ---
## Signif. codes:  0 '***' 0.001 '**' 0.01 '*' 0.05 '.' 0.1 ' ' 1
\end{verbatim}

\begin{Shaded}
\begin{Highlighting}[]
\NormalTok{model.}\FloatTok{2.}\NormalTok{Final <-}\StringTok{ }\KeywordTok{lm}\NormalTok{(DiffinPH}\OperatorTok{~}\NormalTok{Treatment}\DecValTok{-1}\NormalTok{, }\DataTypeTok{data=}\NormalTok{Final.Data) }\CommentTok{#Linear contrasts set up}

\NormalTok{Contrast.Mat<-}\KeywordTok{matrix}\NormalTok{(}\KeywordTok{c}\NormalTok{(}
  \OperatorTok{+}\DecValTok{1}\NormalTok{, }\DecValTok{-1}\NormalTok{, }\DecValTok{0}\NormalTok{, }\DecValTok{0}\NormalTok{,}
  \OperatorTok{+}\DecValTok{0}\NormalTok{, }\DecValTok{1}\NormalTok{, }\DecValTok{-1}\NormalTok{, }\DecValTok{0}\NormalTok{,}
  \OperatorTok{+}\DecValTok{0}\NormalTok{, }\DecValTok{0}\NormalTok{, }\DecValTok{1}\NormalTok{, }\DecValTok{-1}\NormalTok{, }
  \OperatorTok{+}\DecValTok{1}\NormalTok{, }\DecValTok{-2}\NormalTok{, }\DecValTok{1}\NormalTok{, }\DecValTok{0}\NormalTok{, }
  \OperatorTok{+}\DecValTok{1}\NormalTok{, }\DecValTok{0}\NormalTok{, }\DecValTok{1}\NormalTok{, }\DecValTok{-2}\NormalTok{, }
  \OperatorTok{+}\DecValTok{0}\NormalTok{, }\DecValTok{1}\NormalTok{, }\DecValTok{0}\NormalTok{, }\DecValTok{-1}\NormalTok{), }\DataTypeTok{nrow=}\DecValTok{6}\NormalTok{, }\DataTypeTok{byrow=}\OtherTok{TRUE}\NormalTok{)}
\CommentTok{#Linear Contrasts matrix set up}

\KeywordTok{summary}\NormalTok{(}\KeywordTok{glht}\NormalTok{(model.}\FloatTok{2.}\NormalTok{Final, Contrast.Mat), }\DataTypeTok{test=}\KeywordTok{adjusted}\NormalTok{(}\StringTok{"none"}\NormalTok{))}
\end{Highlighting}
\end{Shaded}

\begin{verbatim}
## 
##   Simultaneous Tests for General Linear Hypotheses
## 
## Fit: lm(formula = DiffinPH ~ Treatment - 1, data = Final.Data)
## 
## Linear Hypotheses:
##        Estimate Std. Error t value Pr(>|t|)    
## 1 == 0 -0.76667    0.04509 -17.004  < 2e-16 ***
## 2 == 0  1.51667    0.04509  33.639  < 2e-16 ***
## 3 == 0 -0.45000    0.04509  -9.981 7.09e-13 ***
## 4 == 0 -2.28333    0.07809 -29.239  < 2e-16 ***
## 5 == 0 -0.15000    0.07809  -1.921   0.0613 .  
## 6 == 0  1.06667    0.04509  23.658  < 2e-16 ***
## ---
## Signif. codes:  0 '***' 0.001 '**' 0.01 '*' 0.05 '.' 0.1 ' ' 1
## (Adjusted p values reported -- none method)
\end{verbatim}

\begin{Shaded}
\begin{Highlighting}[]
\CommentTok{#Run the output of complex comparisons}
\end{Highlighting}
\end{Shaded}

\begin{Shaded}
\begin{Highlighting}[]
\NormalTok{tukey.CI<-}\KeywordTok{TukeyHSD}\NormalTok{(}\KeywordTok{aov}\NormalTok{(model.final), }\DataTypeTok{factor=}\NormalTok{Treatment, }\DataTypeTok{conf.level=}\FloatTok{0.95}\NormalTok{)}
\CommentTok{#Tukey CI}
\NormalTok{tukey.CI}
\end{Highlighting}
\end{Shaded}

\begin{verbatim}
##   Tukey multiple comparisons of means
##     95% family-wise confidence level
## 
## Fit: aov(formula = model.final)
## 
## $Treatment
##                                    diff        lwr        upr p adj
## Boiled Water-Control Group    0.7666667  0.6462844  0.8870489 0e+00
## Filtered Water-Control Group -0.7500000 -0.8703823 -0.6296177 0e+00
## Frozen Water-Control Group   -0.3000000 -0.4203823 -0.1796177 2e-07
## Filtered Water-Boiled Water  -1.5166667 -1.6370489 -1.3962844 0e+00
## Frozen Water-Boiled Water    -1.0666667 -1.1870489 -0.9462844 0e+00
## Frozen Water-Filtered Water   0.4500000  0.3296177  0.5703823 0e+00
\end{verbatim}

\begin{Shaded}
\begin{Highlighting}[]
\CommentTok{#Run output of Tukey CI}
\end{Highlighting}
\end{Shaded}

\newpage

\hypertarget{references}{%
\subsection{References}\label{references}}

Keppel, G., \& Wickens, T. D. (2007). \emph{Design and analysis: A
researcher's handbook}. Academic Internet Publishers Incorporated.

Akter, T., Jhohura, F. T., Akter, F., Chowdhury, T. R., Mistry, S. K.,
Dey, D., Barua, M. K., Islam, M. A., \& Rahman, M. (2016). Water quality
index for MEASURING drinking water quality in rural Bangladesh: A
cross-sectional study. \emph{Journal of Health, Population and
Nutrition}, 35(1). \url{https://doi.org/10.1186/s41043-016-0041-5}

Kappler, S., Krahl, M., Geissinger, C., Becker, T., \& Krottenthaler, M.
(2010). Degradation of iso-\(\alpha\)-acids during wort boiling.
\emph{Journal of the Institute of Brewing}, 116(4), 332--338.
\url{https://doi.org/10.1002/j.2050-0416.2010.tb00783.x}

Kim, E. J., Herrera, J. E., Huggins, D., Braam, J., \& Koshowski, S.
(2011). Effect of ph on the concentrations of lead and trace
contaminants in drinking water: A combined batch, pipe loop and sentinel
home study. \emph{Water Research}, 45(9), 2763--2774.
\url{https://doi.org/10.1016/j.watres.2011.02.023}

\end{document}
